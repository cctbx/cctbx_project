\documentclass[12pt, letterpaper]{article}
\usepackage[english]{babel}
\usepackage{amsmath}
\usepackage{amssymb}
\usepackage{accents}

  %%%%%%%%%%%%
  % PREAMBLE %
  %%%%%%%%%%%%
\begin{document}
\selectlanguage{english}

\section{Whiteboard 2013-09-10}

\begin{equation}
  f = \frac{1}{10} \sum_{i} \limits
    ( G  p I_{\mathrm{calc}} + K - I_{\mathrm{obs}})^{2}
\end{equation}
where $g_{i} = G p I_{\mathrm{calc}} + K - I_{\mathrm{obs}}$ is
\texttt{func\_callable}

\begin{equation}
  \frac{\partial g}{\partial \theta_{x}}
    = ( G I_{\mathrm{c}} ) \frac{\partial p}{\partial \theta_{x}}
\end{equation}
where $\partial g / \partial \theta_{x}$ is \texttt{jacobian\_callable}

\begin{equation}
  \frac{\partial f}{\partial \theta_{x}}
    = 2 \sum (G p I_{c} + K - I_{o}) (G I_{c})
        \frac{\partial p}{\partial \theta_{x}}
\end{equation}

\begin{equation}
  \frac{\partial f}{\partial K}
    = 2 \sum_{i} \limits g_{i} \frac{\partial g_{i}}{\partial K}
    = 2 \sum_{i} \limits g_{i}
\end{equation}

\begin{equation}
  \frac{\partial f}{\partial G}
    = 2 \sum_{i} \limits g_{i} \frac{\partial g_{i}}{\partial G}
    = 2 \sum_{i} \limits g_{i} p I_{\mathrm{calc}}
\end{equation}

\begin{equation}
  \frac{\partial g_{i}}{\partial \theta_{x}},
  \frac{\partial g_{i}}{\partial \xi},
  \ldots
\end{equation}

\begin{equation}
  p_{B} = \frac{r_{s}^{2}}{2 r_{h}^{2} + r_{s}^{2}}
\end{equation}

\begin{equation}
  \frac{\partial p_{B}}{\partial \theta_{x}}
    = \frac{-4 r_{s}^{2} r_{h}}{(2 r_{h}^{2} + r_{s}^{2})^{2}}
      \frac{\partial r_{h}}{\partial \theta_{x}}
\end{equation}

\begin{equation}
  r_{h} = | \underaccent{\tilde}{S} |  - \frac{1}{\lambda}
\end{equation}

\begin{equation}
  r_{h} = (\underaccent{\tilde}{S} \cdot \underaccent{\tilde}{S})^{1 / 2} -
          \frac{1}{\lambda}
\end{equation}

\begin{equation}
  \frac{\partial r_{h}}{\partial \theta_{x}}
    = \frac{1}{2}
      \frac{1}{\sqrt{\underaccent{\tilde}{S} \underaccent{\tilde}{S}}}
      \frac{\partial \underaccent{\tilde}{S} \underaccent{\tilde}{S}}
           {\partial \theta_{x}}
\end{equation}

\begin{equation}
  \frac{\partial \underaccent{\tilde}{S} \underaccent{\tilde}{S}}
       {\partial \theta_{x}}
    = 2 \underaccent{\tilde}{S}
      \frac{\partial \underaccent{\tilde}{S}}{\partial \theta_{x}}
\end{equation}

\begin{equation}
  \underaccent{\tilde}{S} = \underaccent{\tilde}{x} + S_{0}
\end{equation}


\begin{equation}
  \frac{\partial \underaccent{\tilde}{S}}{\partial \theta_{x}}
    = \frac{\partial \underaccent{\tilde}{x}}{\partial \theta_{x}}
\end{equation}

\begin{equation}
  \underaccent{\tilde}{x} = A^{*} h
\end{equation}

\begin{equation}
  A^{*} = \mathbb{R}_{y} ( \mathbb{R}_{x} ( A_{0}^{*} ) )
\end{equation}

Aside:
\begin{equation}
  \frac{\partial A h}{\partial u}
    = A \frac{\partial h}{\partial u} +
      \frac{\partial A}{\partial u} h
\end{equation}

\begin{equation}
  \frac{\partial \underaccent{\tilde}{x}}{\partial \theta_{x}}
    = A^{*} \frac{\partial}{\partial \theta_{x}} \underaccent{\tilde}{h} +
      \frac{\partial A^{*}}{\partial \theta_{x}} \underaccent{\tilde}{h}
\end{equation}
where $\partial \underaccent{\tilde}{h} / \partial \theta_{x} = 0$ and
$\partial A^{*} / \partial \theta_{x}$ corresponds to
\texttt{dA\_drotx = X}, $( \underaccent{\tilde}{h}, k, l )$

\begin{equation}
  \frac{\partial A^{*}}{\partial \theta_{x}}
    = \mathbb{R}_{y} \frac{\partial R_{x} A_{0}^{*}}{\partial \theta_{x}} +
      \frac{\partial R_{y}}{\partial \theta_{x}} R_{x} A_{0}^{*}
\end{equation}
where  $\partial R_{y} / \partial \theta_{x} = 0$

\begin{equation}
  \frac{\partial R_{x} A_{0}^{*}}{\partial \theta_{x}}
    = R_{x} \frac{\partial A_{0}^{*}}{\partial \theta_{x}} +
      \frac{\partial R_{x}}{\partial \theta_{x}} A_{0}^{*}
\end{equation}
where $\partial A_{0}^{*} / \partial \theta_{x} = 0$ and $\partial
R_{x} / \partial \theta_{x}$ in the code.


\begin{equation}
  \frac{\partial A^{*}}{\partial \theta_{y}}
    = \mathbb{R}_{y} \frac{\partial R_{x} A_{0}^{*}}{\partial \theta_{y}} +
      \frac{\partial R_{y}}{\partial \theta_{y}} (R_{x} A_{0}^{*})
\end{equation}
where $\partial R_{x} A_{0}^{*} / \partial \theta_{y} = 0$

\begin{equation}
  \frac{\partial A^{*}}{\partial \theta_{y}}
    = \frac{\partial R_{y}}{\partial \theta_{x}} (R_{x} A_{0}^{*})
\end{equation}

\begin{equation}
  \frac{\partial A^{*}}{\partial \theta_{x}}
    = [R_{y}] \left[ \frac{\partial R_{x}}{\partial \theta_{x}} A_{0}^{*} \right]
\end{equation}



\end{document}
