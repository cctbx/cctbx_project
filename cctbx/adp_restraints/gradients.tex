\documentclass[11pt]{article}
\usepackage{cctbx_preamble}
\usepackage{amscd}

\title{Restraint Gradients}
\author{\rjgildea}
\date{\today}

\begin{document}
\maketitle

\section{ADP similarity restraint}
The anisotropic displacement parameters of two atoms are restrained to have the
same $U_{ij}$ components.  This is equivalent to a SHELXL SIMU restraint \cite{SHELX:man97}.
The weighted least-squares residual is
\begin{equation}
R = w \sum_{i=1}^3 \sum_{j=1}^i (U_{A,ij} - U_{B,ij})^2,
\end{equation}
and the gradient of the residual with respect to the component $U_{A,ij}$ is then
\begin{equation}
\partialder{R}{U_{A,ij}} = 2w(U_{A,ij} - U_{B,ij}).
\end{equation}

\section{Rigid-bond restraint}

In a `rigid-bond' restraint the components of the anisotropic displacement parameters
of two atoms in the direction of the vector connecting those two atoms are restrained
to be equal.  This corresponds to Hirshfeld's `rigid-bond' test \cite{Hirshfeld:1976} for testing
whether anisotropic displacement parameters are physically reasonable (see SHELX
manual, DELU restraint \cite{SHELX:man97}).  We must therefore minimise the mean square displacement of
the atom in the direction of the bond.

The weighted least-squares residual is then
\begin{equation}
R = w(z^2_{A,B} - z^2_{B,A})^2,
\end{equation}
where in the Cartesian coordinate system the mean square displacement of atom A
along the vector $\overrightarrow{AB}$, $z^2_{A,B}$, is given by
\begin{equation}
z^2_{A,B} = \frac{\vec{r}^t\mat{U}_{cart,A}\vec{r}}{\norm{\vec{r}}^2},
\end{equation}
where
\begin{equation}
\vec{r} = \begin{pmatrix} x_A - x_B\\y_A - y_B\\z_A - z_B \end{pmatrix}
= \begin{pmatrix} x\\y\\z \end{pmatrix},
\end{equation}
$\vec{r}^t$ is the transpose of $\vec{r}$ (\textit{i.e.} a row vector) and
$\norm{\vec{r}}$ is the length of the vector $\overrightarrow{AB}$.

The derivative of the residual with respect to an element of $\vec{U}_{cart,A}$,
$U_{A,ij}$ is given by (using the chain rule)
\begin{align}
\partialder{R}{U_{A,ij}} &= \partialder{R}{z^2_{A,B}} \partialder{z^2_{A,B}}{U_{A,ij}}\\
&=2w(z^2_{A,B} - z^2_{B,A}) \partialder{z^2_{A,B}}{U_{A,ij}}\label{eqn:r_derivative}
\end{align}

The matrix multiplication in obtaining $z^2_{A,B}$ can be evaluated as follows
(remembering $\vec{U}_{cart}$ is symmetric):
\begin{align}
\vec{r}^t\vec{U}_{cart,A}\vec{r} &= 
\begin{pmatrix} x & y & z \end{pmatrix}
\begin{pmatrix} U_{11} & U_{12} & U_{13}\\
                         U_{12} & U_{22} & U_{23}\\
                         U_{13} & U_{23} & U_{33}\end{pmatrix}
\begin{pmatrix} x\\y\\z\end{pmatrix}\\
&= U_{11}\ x^2 + U_{22}\ y^2 + U_{33}\ z^2 + 2U_{12}\ xy + 2U_{13}\ xz + 2U_{23}\ yz
\end{align}
It then follows that
\begin{equation}
\partialder{z^2_{A,B}}{U_{11}} = \frac{x^2}{\norm{\vec{r}}^2} ,\qquad
\partialder{z^2_{A,B}}{U_{22}} = \frac{y^2}{\norm{\vec{r}}^2} ,\qquad
\partialder{z^2_{A,B}}{U_{33}} = \frac{z^2}{\norm{\vec{r}}^2} ,
\end{equation}
and
\begin{equation}
\partialder{z^2_{A,B}}{U_{12}} = \frac{2xy}{\norm{\vec{r}}^2} ,\qquad
\partialder{z^2_{A,B}}{U_{13}} = \frac{2xz}{\norm{\vec{r}}^2} ,\qquad
\partialder{z^2_{A,B}}{U_{23}} = \frac{2yz}{\norm{\vec{r}}^2} .
\end{equation}
These can be combined with \eqnref{r_derivative} to give us the derivatives
with respect to each $U_{ij}$ component.

\bibliography{cctbx_references}

\end{document}