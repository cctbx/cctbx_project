\documentclass[11pt]{article}
\usepackage{geometry}                % See geometry.pdf to learn the layout options. There are lots.
\geometry{letterpaper}                   % ... or a4paper or a5paper or ... 
%\geometry{landscape}                % Activate for for rotated page geometry
%\usepackage[parfill]{parskip}    % Activate to begin paragraphs with an empty line rather than an indent
\usepackage{graphicx}
\usepackage{amsmath}
\usepackage{epstopdf}
\usepackage{cctbx_notations}
\DeclareGraphicsRule{.tif}{png}{.png}{`convert #1 `dirname #1`/`basename #1 .tif`.png}

\title{Transform / Change of Basis\\of\\Electron Density / Structure Factors}
\author{\lucjbourhis}
\date{\today}                                           % Activate to display a given date or no date

\begin{document}
\maketitle

\section{Transform}

Let us consider an electron density $\rho(x)$ and an affine operator $\sym{R}{t}$ transforming $x$. Then the transform of the former by the latter is $\rho(\sym{R}{t}^{-1}x)$. This is a classic if at first counter-intuitive result. The transform of $\rho$ must indeed take the value $\rho(x)$ at the point $\sym{R}{t}x$ (most easily visualised in 1 dimension on the special case of a translation $x \mapsto x+\tau$ by looking at the graph of $\rho$: that graph is moved along the $x$-axis by a amount $\tau$).

Borrowing the notation from group theory, we will denote the transform of $\rho$ by
\begin{equation}
\rho^{\sym{R}{t}}(x) = \rho(\sym{R}{t}^{-1}x)
\label{rho:transform}
\end{equation}

Let us now find the equivalent of eq.~(\ref{rho:transform}) for the Fourier transform $F$ of $\rho$. From now on, we will only consider the following case: $\sym{R}{t}$ leaves the unit cell $U$ invariant.

Then\footnote{It is most natural in crystallography to represent any position $x$ by a column vector and any miller index $h$ by a row vector (mathematically speaking $h$ is in the reciprocal space of $x$ and therefore $h$ is really a linear form). Therefore the scalar product $h.x$ reads like the mere matrix product $hx$ and the operator relation $h.Rx = R^Th.x$ is the trivial matrix product $hRx$ interpreted in two ways using associativity.}
\begin{align}
F^{\sym{R}{t}}(h) &= \int_U \rho(\sym{R}{t}^{-1}x) e^{i2\pi hx} d^3x ; \nonumber \\
&= \int_{\sym{R}{t}U} \rho(y) e^{i2\pi (hRx + ht)} d^3x ;\nonumber\\
\intertext{with the change of variable $y = \sym{R}{t}^{-1}x$ and then}
F^{\sym{R}{t}}(h) &= F(hR) e^{i2\pi ht}
\end{align}
with the invariance of $U$.

The \cctbx\ does not implement such transforms of the structure factors. But it implements the other side of the coin: changes of basis.

\section{Change of basis}

Let us consider the change of basis transforming the old basis\footnote{We should actually be talking about a frame, i.e. an origin $\omega$ and a basis of vectors $(e_1, e_2, e_3)$ but the \cctbx\ calls it a basis.} of real space $(\omega, e_1, e_2, e_3)$ into the new basis $(\omega', e'_1, e'_2, e'_3)$ through the operator $\sym{R}{t}$. A well-known result is that if $x$ is the coordinate vector of a position w.r.t. $(e_1, e_2, e_3)$ and $x'$ the coordinate vector of the same position w.r.t. $(e'_1, e'_2, e'_3)$, then\footnote{Here, by an abuse of notation, we identify the operator on the vectors $e_i$ and the operator on the coordinate vector $x$.}
\begin{equation}
x = \sym{R}{t}x'.
\label{change::of::basis::position}
\end{equation}
Reminder:
\begin{align}
\omega' + \sum_j x'_j e'_j = \omega + t + \sum_j x'_j \sum_i R_{ij} e_i = \omega + \sum_i (\underbrace{t_i + \sum_j R_{ij} x'_j}_{x_i \text{ by definition}})e_i. \nonumber
\end{align}
A corollary is the corresponding law for the miller indices,
\begin{equation}
h' = h\sym{R}{t}.
\end{equation}
Indeed, since scalar product should be independent of the basis, $h'$ must statisfy $h'\Delta x' = h\Delta x$ for any vector $\Delta x'$ (as opposed to the point $x$\footnote{Let's not forget the positions make an affine space whereas the miller indices are only the reciprocal of the associated vector space and of course eq.~(\ref{change::of::basis::position}) results in $\Delta x = R \Delta x'$}). The \cctbx\ implements this formula in C++ by operator overloading.

In the previous section, we considered transforms where the old $x$ is moved onto the new $x'$ as $x' = \sym{R}{t} x$. Thus a change of basis $\sym{R}{t}$ corresponds to a transform $\sym{R}{t}^{-1}$. As a result, the \cctbx\ code handling a change of basis can trivially be used to generate a transform by simply passing as the change-of-basis operator the inverse of the transform we want.

Let us now work out the formula relating the structure factors before and after the change of basis. If $F(h)$ is the structure factor in the old basis, we will denote the structure factor in the new basis as $F'(h)$. With the remark in the previous paragraph,
\begin{align}
F'(h) &= F^{\sym{R}{t}^{-1}}(h) = F(hR^{-1}) e^{i2\pi h(-R^{-1}t)}\nonumber\\
\intertext{and therefore}
F'(hR) &= F(h)e^{-i2\pi ht}.
\end{align}
That is the formula implemented in the \cctbx\ in \code{sym\_equiv.h}, c.f. \code{sym\_equiv\_index\linebreak[0]::phase\_eq} and it's use in \code{change\_basis.h}. It is particularly convenient since a \code{miller\linebreak[0].array} stores $h$ and $F(h)$ in two parallel arrays. By looping over the both of them at the same time, one can compute and immediately store the new miller index $hR$ and the value $F'$ for that new miller index.

\section{Symmetry cross-correlation: an application}

The following questions is recurrent after any method processing a structure in P1: is the structure invariant under an operator $\sym{R}{t}$ in some basis which is not necessarily the one we have done that processing in? That is true of the dual space solution method (Phenix.hyss, ShelXD) and also of charge flipping.

To be continued.

\end{document}  